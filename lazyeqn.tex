\documentclass[DIV15]{scrartcl}
\usepackage{lazyeqn}
\usepackage{booktabs}
\makeatletter
% \texttt{\textbackslash\expandafter\@gobble\string#1}
% \newcommand{\showcasemath}[1]{\expandafter\verb+#1+ & $#1$}
\makeatother


\makeatletter
\newcommand{\myverb}{%
    \begingroup
    % deactivate special characters
    \let\do\@makeother
    \dospecials
    % change '{' and '}' back to normal
    \catcode`\{=1
    \catcode`\}=2
    \@myverb%
}
\def\@myverb#1{%
    \endgroup%
    \texttt{#1}%
}
% \newcommand{\scm}[3]{%
%   % \vspace{2.1ex}
%   \noindent
%   \def\tmpp{#1}%
%   \def\tmp{#3}\fbox{%
%   \begin{minipage}{\linewidth}\vspace{0.5ex}
%   \begin{minipage}[t]{0.17\linewidth}
%     \texttt{\expandafter\foo\meaning\tmpp}
%   \end{minipage}
%   \begin{minipage}[t]{0.37\linewidth}%
%     #2
%   \end{minipage}
%   \begin{minipage}[t]{0.24\linewidth}%
%     \texttt{\expandafter\foo\meaning\tmp}
%   \end{minipage}
%   \begin{minipage}[t]{0.20\linewidth}
%     #3
%   \end{minipage}\vspace{0.5ex}
%   \end{minipage}}\newline\noindent
% }

\newcommand{\scm}[3]{%
  % \vspace{2.1ex}
  \noindent
  \def\tmpp{#1}%
  \def\tmp{#3}%
  \begin{minipage}{\linewidth}\vspace{0.5ex}
  \begin{minipage}[t]{0.17\linewidth}
    \texttt{\expandafter\foo\meaning\tmpp}
  \end{minipage}
  \begin{minipage}[t]{0.37\linewidth}%
    #2
  \end{minipage}
  \begin{minipage}[t]{0.24\linewidth}%
    \texttt{\expandafter\foo\meaning\tmp}
  \end{minipage}
  \begin{minipage}[t]{0.20\linewidth}
    #3
  \end{minipage}\vspace{0.5ex}
  \end{minipage}\newline\noindent
}

\makeatother
\def\foo#1>{}
\makeatletter
\newcommand{\showcasemath}{%
    \begingroup
    % deactivate special characters
    \let\do\@makeother
    \dospecials
    % change '{' and '}' back to normal
    \catcode`\{=1
    \catcode`\}=2
    \@myverb%
}
\makeatother

\title{\texttt{lazyeqn}\textnormal{ --- equation macros for \LaTeXe}}
\author{H.O. Solmaz}
\begin{document}
\maketitle

\section{Introduction}

People who use \LaTeX{} everyday eventually realize that vanilla \LaTeX{} with
amsmath and similar packages is not sufficiently rich with operators or
symbols to enable abstraction that is necessary for typing fast---at least not
fast enough. Most of these people realize that their typing involves a lot of
repetition which can be reduced through abstraction. They define macros for
symbols, operators, and anything that can be thought up which makes sense in the
formulatory context.

It cannot be expected from amsmath---which is the primary package for
typesetting math---to include such \emph{contextual macros}. Its purpose is to
constitute the robust and elegant basis for which the contextual macros are
supposed to be built upon; and that it succeeds in. Nevertheless I have
witnessed the acts of a lot of colleagues of mine who are not aware of such
capabilities, and do everything the hard, vanilla way.

Surely, independent of their disciplines, Physicists, Chemists, Engineers,
Mathematicians, Computer Scientists etc. all use some sorts of mathematics.
Specifically, they solve equation systems, differential equations, algebraic
equations of different sorts, which all require utilization of different symbols
in individual contexts. These disciplines often intersect, and can benefit from
a common package which shall provide all the symbols required by respective
contexts. \texttt{lazyeqn} aims to be that package. It is about standardization and
comfort. It reduces the number of characters you are supposed to type by 50\%.

Check out the examples section to see how using \texttt{lazyeqn} can simplify
your life.
\section{Examples}

Let's start with a simple and minimalist example.
Poisson's equation in Cartesian coordinates is written as
%
\begin{equation}
  \left( \frac{\partial^2}{\partial x^2} + \frac{\partial^2}{\partial y^2} +
  \frac{\partial^2}{\partial z^2} \right)\varphi(x,y,z) = f(x,y,z).
\end{equation}
%
I took this example directly from Wikipedia, where the MediaWiki plugin
\texttt{texvc} validates the equation using a weird parser written in OCaml,
creates a \LaTeX{} file which contains only that equation and compiles it into a
PNG. I explain all this, because there are no options to use any packages for
equations in Wikipedia, and everything must be vanilla amsmath. As such, the
equation above is typeset with the following code
%
\begin{verbatim}
\begin{equation}
  \left(\frac{\partial^2}{\partial x^2}+\frac{\partial^2}{\partial y^2}+
  \frac{\partial^2}{\partial z^2}\right)\varphi(x,y,z)=f(x,y,z)
\end{equation}
\end{verbatim}
%
Now, with \texttt{lazyeqn}, it becomes
%
% \begin{equation}
%   \rbr{\partdd{}{x}+\partdd{}{y} + \partdd{}{z}}\varphi(x,y,z) = f(x,y,z)
% \end{equation}
%
\begin{verbatim}
\begin{equation}
  \rbr{\partdd{}{x}+\partdd{}{y}+\partdd{}{z}}\varphi(x,y,z)=f(x,y,z)
\end{equation}
\end{verbatim}
%
To compare, vanilla amsmath description has 134 characters, and with
\texttt{lazyeqn} the number reduces to 72. That is almost a \textbf{50\% reduction}!

Moving onto an example which uses letter symbols (like bold, calligraphic etc.).
The symmetric fourth-order identity tensor is written as
%
\begin{equation}
  \BFI^{(s)} =\half(\delta_{ik}~\delta_{jl}+\delta_{il}~\delta_{jk})~\basis{ijkl}
\end{equation}
%
With vanilla amsmath, it is
\begin{verbatim}
\begin{equation}
  \boldsymbol{\mathsf{I}}^{(s)}=\frac{1}{2}~(\delta_{ik}~\delta_{jl}+
  \delta_{il}~\delta_{jk})~\mathbf{e}_i\otimes\mathbf{e}_j\otimes
  \mathbf{e}_k\otimes\mathbf{e}_l
\end{equation}
\end{verbatim}
%
However with \texttt{lazyeqn} it becomes
%
\begin{verbatim}
\begin{equation}
  \BFI =\half(\delta_{ik}~\delta_{jl}+\delta_{il}~\delta_{jk})~\basis{ijkl}
\end{equation}
\end{verbatim}
%
The reduction ratio this time is 79 to 164, which is 51\% reduction in number of
characters. It is arguable that time is vary valuable, but with
\texttt{lazyeqn}, you can type twice as fast, without taking into account
the additional time required to debug a code twice the size it should be.
\section{Symbol Macros}

This package aims to make typing of symbols much easier by providing some macros
prefixing conventions for different math alphabets. For example to type in a
upright bold A, one can use the \verb+\VA+ macro which prefixes the letter A
with the macro character \verb+\+ and the letter \verb+V+, which has been chosen
as the convention for ``upright bold'' math symbols.

These conventions have been chosen to reduce the number of characters typed,
whereas also to make sure the equations stay readable in code. For example, one might guess
the letter B might stand for bold and I might stand for italic in some prefixes,
but the prefixes don't have to make sense for all alphabets.
For the list of all prefixes for
specific alphabets, please refer to Table~\ref{tb:math}.

Normal Greek letters are slanted in lowercase and upright in uppercase. To
obtain bold Greek letters, use the \verb+\B+ prefix, e.g. \verb+\BXi+ for
uppercase xi and \verb+\Bxi+ for lowercase xi.

Also, \verb+\vareps+ and \verb+\varthe+ macros are defined as shortcuts to \verb+\varepsilon+ and \verb+\vartheta+

For numbers, the written forms can be prefixed to obtain corresponding symbols,
like \verb+\Vone+ and \verb+\Izero+. Observe the samples in Table~\ref{tb:math}
for alphabets in which this is possible.


\begin{table}[htbp]
  \centering
\begin{tabular}{p{14em} l l l}
  \toprule
  Name & \LaTeX{} package & Prefix & Sample \\
  \midrule
  Italic letters & none & none & $ABCdef$ \\
  Upright letters & none & \verb+\U+ & $\UA\UB\UC\Ud\Ue\Uf$\\
  Italic bold letters & none & \verb+\B+ & $\BA\BB\BC\Bd\Be\Bf\Bone\Btwo\Bthree$ \\
  Upright bold letters & none & \verb+\V+ & $\VA\VB\VC\Vd\Ve\Vf\Vone\Vtwo\Vthree$ \\
  Sans serif & none & \verb+\F+ & $\FA\FB\FC\Fd\Fe\Ff$\\
  Bold sans serif & none & \verb+\BF+ & $\BFA\BFB\BFC\BFd\BFe\BFf$\\
  Calligraphic letters (uppercase only) & none & \verb+\C+ & $\CA\CB\CC$ \\
  Calligraphic bold letters (uppercase only) & none & \verb+\BC+ & $\BCA\BCB\BCC$ \\
  AMS blackboard bold (uppercase only) & \texttt{amsfonts} & \verb+\I+ & $\IA\IB\IC$ \\
  \texttt{bbold} blackboard bold & \texttt{bbold},\texttt{mathbbol} & \verb+\II+ & $\IIA\IIB\IIC\IId\IIe\IIf\IIone\IItwo\IIthree$ \\
  \texttt{bbm} blackboard bold sans & \texttt{bbm} & \verb+\BB+ & $\BBA\BBB\BBC\BBd\BBe\BBf$ \\
  \texttt{bbm} blackboard bold serif & \texttt{bbm} & \verb+\bB+ & $\bBA\bBB\bBC\bBd\bBe\bBf$\\
  \texttt{bbm} blackboard bold sans italic & \texttt{bbm} & \verb+\IB+ & $\IBA\IBB\IBC\IBd\IBe\IBf$\\
  \texttt{bbm} blackboard bold serif italic & \texttt{bbm} & \verb+\iB+ & $\iBA\iBB\iBC\iBd\iBe\iBf$\\
  Ralph Smith’s Formal Script (uppercase only) & \texttt{rsfs} & \verb+\ral+ & $\ralA\ralB\ralC$\\
  Ralph Smith’s Formal Script bold (uppercase only) & \texttt{rsfs} & \verb+\Bral+ & $\BralA\BralB\BralC$\\
  \texttt{euscript} cursive (uppercase only) & \texttt{euscript} & \verb+\eul+ & $\eulA\eulB\eulC$\\
  \texttt{euscript} cursive bold (uppercase only) & \texttt{euscript} & \verb+\Beul+ & $\BeulA\BeulB\BeulC$\\
  Fraktur & \texttt{yfonts} & \verb+\frak+ & $\frakA\frakB\frakC\fraka\frakb\frakc$\\
  Fraktur bold & \texttt{yfonts} & \verb+\Bfrak+ & $\BfrakA\BfrakB\BfrakC\Bfraka\Bfrakb\Bfrakc$\\
  \bottomrule
\end{tabular}
\caption{Guide for symbol prefixes for latin letters. For more information about math alphabets,
  please refer to {\em Table 213 -- Math Alphabets} of The
  Comprehensive \LaTeX{} Symbol List by Scott Pakin.}
\label{tb:math}
\end{table}

\section{Operator Macros}

\subsection{Braces}

% \scm{\rbr}{Round braces}{$\rbr{\VA}$}
\scm{\rbr}{Round braces - adaptive}{$\rbr{\VA}$}
\scm{\rbrn}{Round braces - normal sized}{$\rbrn{\VA}$}
\scm{\rbrbig}{Round braces - big}{$\rbrbig{\VA}$}
\scm{\rbrBig}{Round braces - Big}{$\rbrBig{\VA}$}
\scm{\rbrbigg}{Round braces - bigg}{$\rbrbigg{\VA}$}
\scm{\rbrBigg}{Round braces - Bigg}{$\rbrBigg{\VA}$}
%
\scm{\sbr}{Square brackets - adaptive}{$\sbr{\VA}$}
\scm{\sbrn}{Square brackets - normal sized}{$\sbrn{\VA}$}
\scm{\sbrbig}{Square brackets - big}{$\sbrbig{\VA}$}
\scm{\sbrBig}{Square brackets - Big}{$\sbrBig{\VA}$}
\scm{\sbrbigg}{Square brackets - bigg}{$\sbrbigg{\VA}$}
\scm{\sbrBigg}{Square brackets - Bigg}{$\sbrBigg{\VA}$}
%
\scm{\cbr}{Curly braces - adaptive}{$\cbr{\VA}$}
\scm{\cbrn}{Curly braces - normal sized}{$\cbrn{\VA}$}
\scm{\cbrbig}{Curly braces - big}{$\cbrbig{\VA}$}
\scm{\cbrBig}{Curly braces - Big}{$\cbrBig{\VA}$}
\scm{\cbrbigg}{Curly braces - bigg}{$\cbrbigg{\VA}$}
\scm{\cbrBigg}{Curly braces - Bigg}{$\cbrBigg{\VA}$}
%
% \scm{\abr}{Angles}{$\abr{\VA}$}
\scm{\abr}{Angles - adaptive}{$\abr{\VA}$}
\scm{\abrn}{Angles - normal sized}{$\abrn{\VA}$}
\scm{\abrbig}{Angles - big}{$\abrbig{\VA}$}
\scm{\abrBig}{Angles - Big}{$\abrBig{\VA}$}
\scm{\abrbigg}{Angles - bigg}{$\abrbigg{\VA}$}
\scm{\abrBigg}{Angles - Bigg}{$\abrBigg{\VA}$}

\scm{\dbr}{Double square braces - adaptive}{$\dbr{\VA}$} % maybe use rrbracket-llbracket from stmaryd?
\scm{\dbrl}{Left double square bracket}{$\dbrl$}
\scm{\dbrr}{Right double square bracket}{$\dbrr$}
%
\subsection{Operators}

% \begin{table}[H]
%   \centering
%   \begin{tabular}{l l l l}
%     \toprule
%     Macro & Explanation & Example usage & Result \\
%     \midrule
%     \verb+\tra+ & Transpose & &  \\
%     \verb+\variation{}{}+ & Variation & \verb+\variation{a}{b}+ & $\variation{a}{b}$  \\
%     \verb+\partd{}{}+ & Partial derivative & & \\
%     \verb+\partdb{}{}+ & Partial derivative with brackets & & \\
%     \verb+\cpartd{}{}+ & Partial derivative using \verb+\cfrac+ & & \\
%     \bottomrule
%   \end{tabular}
% \end{table}

% \scm{Macro}{Explanation}{Example usage}
%
\scm{\transposesymbol}{Transpose symbol}{$\transposesymbol$}
\scm{\tra}{Transpose}{$\VA\tra$}
% \scm{\tra[]}{Transpose with preceding argument}{$\tra[-]$}
\scm{\trab}{Transpose with preceding argument}{$\VA\trab{-}$}
\scm{\traf}{Transpose with following argument}{$\VA\traf{-1}$}
\scm{\trabf}{Transpose with preceding and following argument}{$\VA\trabf{-}{-1}$}
\scm{\trat}{Transpose with argument on top}{$\VA\trat{23}$}
\scm{\inv}{Inverse}{$\VA\inv$}
\scm{\invtra}{Invert, transpose}{$\VA\invtra$}
\scm{\trainv}{Transpose, invert}{$\VA\trainv$}
\scm{\dif}{Differential operator d}{$\dif a$}
\scm{\del}{Partial derivative operator}{$\del a$}
\scm{\var}{Variational operator d}{$\var a$}
\scm{\vartn}{Variation}{$\vartn{a}{b}$}
\scm{\partd}{Partial derivative}{$\partd{a}{b}$}
\scm{\partdd}{Double partial derivative}{$\partdd{a}{b}$}
\scm{\epartd}{Euler type partial derivative}{$\epartd{a}{b}$}
\scm{\deriv}{Derivative}{$\deriv{a}{b}$}
\scm{\derivd}{Double derivative}{$\derivd{a}{b}$}
\scm{\ederiv}{Euler type derivative}{$\ederiv{a}{b}$}
%
\scm{\ddt}{Time derivative}{$\ddt x$}
\scm{\ddelt}{Partial Time derivative}{$\ddelt x$}
%
\scm{\ddx}{Arbitrary derivative}{$\ddx a$}
\scm{\ddelx}{Arbitrary partial derivative}{$\ddelx a$}
%
\scm{\assem}{Assembly symbol}{$\assem$}
\scm{\evat}{Evaluated at}{$a\evat_{b=0}$}
\scm{\degree}{Degree symbol}{\degree{}C}
\scm{\suml}{Summation made easier, with limits}{$\suml{i=1}{3}$}
\scm{\sumn}{Summation made easier, with nolimits}{$\sumn{i=1}{3}$}
\scm{\dtp}{Dot product (\texttt{cdot} shortcut)}{$\VA\dtp\VB$}
\scm{\crs}{Cross product (\texttt{times} shortcut)}{$\VA\crs\VB$}
\scm{\dcrs}{Double cross product }{$\VA\dcrs\VB$}
\scm{\dyc}{Dyadic product (\texttt{otimes} shortcut)}{$\VA\dyc\VB$}
\scm{\linedot}{Time integral with overline}{$\linedot{\UA}$}
\scm{\DIV}{DIVergence}{$\DIV\VA$}
\scm{\Div}{Divergence}{$\Div\VA$}
\scm{\div}{divergence}{$\div\VA$}
\scm{\curl}{Curl}{$\curl\VA$}
\scm{\rot}{Rotation}{$\rot\VA$}
\scm{\proj}{Projection}{$\proj\VA$}
\scm{\sign}{Sign function}{$\sign\VA$}
\scm{\unit}{Unit function}{$\unit\VA$}
\scm{\arg}{arg}{$\arg\VA$}
\scm{\Arg}{Arg}{$\Arg\VA$}
\scm{\sym}{Symmetric}{$\sym\VA$}
\scm{\TR}{TRace}{$\TR\VA$}
\scm{\tr}{trace}{$\tr\VA$}
\scm{\trial}{Trial function}{$\trial\VA$}
\scm{\DEV}{DEViatoric}{$\DEV\VA$}
\scm{\dev}{deviatoric}{$\dev\VA$}
\scm{\GRAD}{GRADient}{$\GRAD\VA$}
\scm{\Grad}{Gradient}{$\Grad\VA$}
\scm{\grad}{gradient}{$\grad\VA$}
\scm{\Lin}{Linearization}{$\Lin\VA$}
\scm{\lin}{linearization}{$\lin\VA$}
\scm{\skw}{Skew}{$\skw\VA$}
\scm{\loc}{loc}{$\loc\VA$}
\scm{\cof}{Cofactor}{$\cof\VA$}
\scm{\reac}{reac}{$\reac\VA$}
\scm{\pind}{pind}{$\pind\VA$}
\scm{\ndim}{ndim}{$\ndim$}
\scm{\diag}{diag}{$\diag\VA$}
\scm{\veci{}}{Indexed list representation}{$\veci{a}$}
\scm{\dV}{Volume element}{$\dV$}
\scm{\dv}{Volume element}{$\dv$}
\scm{\dA}{Area element}{$\dA$}
\scm{\da}{Area element}{$\da$}
\scm{\dS}{Surface element}{$\dS$}
\scm{\dx}{Line element}{$\dx$}
\scm{\dt}{Time difference}{$\dt$}
\scm{\Abs{}}{Absolute value}{$\Abs{\VA}$}
\scm{\Norm{}}{Norm}{$\Norm{\VA}$}
\scm{\basis{}}{Vector bases}{$\basis{ijkl}$}
\scm{\rbdot}{Operands}{$\div\rbdot$}
\scm{\so}{Special orthogonal group}{$\so$}
% \scm{\rve}{}{$\rve$}
% \scm{\rpg}{}{$\rpg$}
% \scm{\ehs}{}{$\ehs$}
% \scm{\mo}{}{$\mo$}
\scm{\othree}{Orthogonal group 3}{$\othree$}
\scm{\sothree}{Special orthogonal group 3}{$\sothree$}
\scm{\slthree}{Special linear group 3}{$\slthree$}
\scm{\eqand}{and between equations}{$a \eqand b$}
\scm{\eqor}{and between equations}{$a \eqor b$}
\scm{\eqwith}{with between equations}{$a \eqwith b$}
\scm{\eqcomma}{Comma between equations}{$a\eqcomma b$}
\scm{\eqsemi}{Semicolon between equations}{$a\eqsemi b$}
\scm{\quadd}{Wraps the parameters with quads}{$a\quadd{\rightarrow} b$}

\subsection{Fractions, roots, roman numerals}
\scm{\half}{Half}{$\half$}
\scm{\third}{Third}{$\third$}
\scm{\fourth}{Fourth}{$\fourth$}
\scm{\fifth}{Fifth}{$\fifth$}
\scm{\sixth}{Sixth}{$\sixth$}
\scm{\seventh}{Seventh}{$\seventh$}
\scm{\eighth}{Eighth}{$\eighth$}
\scm{\ninth}{Ninth}{$\ninth$}
\scm{\tenth}{Tenth}{$\tenth$}
\scm{\twothree}{Twothree}{$\twothree$}
\scm{\threetwo}{Threetwo}{$\threetwo$}
\scm{\sqonetwo}{Sqonetwo}{$\sqonetwo$}
\scm{\sqonethree}{Sqonethree}{$\sqonethree$}
\scm{\sqtwothree}{Sqtwothree}{$\sqtwothree$}
\scm{\sqthreetwo}{Sqthreetwo}{$\sqthreetwo$}
\scm{\Rone}{Roman 1}{$\Rone$}
\scm{\Rtwo}{Roman 2}{$\Rtwo$}
\scm{\Rthree}{Roman 3}{$\Rthree$}
\scm{\Rfour}{Roman 4}{$\Rfour$}
\scm{\Rvarfour}{Roman var 4}{$\Rvarfour$}
\scm{\Rfive}{Roman 5}{$\Rfive$}
\scm{\Rsix}{Roman 6}{$\Rsix$}
\scm{\Rseven}{Roman 7}{$\Rseven$}
\scm{\Reight}{Roman 8}{$\Reight$}
\scm{\Rnine}{Roman 9}{$\Rnine$}
\scm{\Rten}{Roman 10}{$\Rten$}

\subsection{Vectors}
\scm{\cvec{}}{Column vector}{$\cvec{1;2;3;4}$}
\scm{\vcvec{}}{Column vector with straight braces}{$\vcvec{1;2;3;4}$}
\scm{\rvec{}}{Row vector}{$\rvec{1;2;3;4}$}
\scm{\vrvec{}}{Row vector with straight braces}{$\vrvec{1;2;3;4}$}

\end{document}
