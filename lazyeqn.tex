\documentclass[DIV15]{scrartcl}
\usepackage{lazyeqn}
\usepackage{booktabs}
\makeatletter
% \texttt{\textbackslash\expandafter\@gobble\string#1}
% \newcommand{\showcasemath}[1]{\expandafter\verb+#1+ & $#1$}
\makeatother


\makeatletter
\newcommand{\myverb}{%
    \begingroup
    % deactivate special characters
    \let\do\@makeother
    \dospecials
    % change '{' and '}' back to normal
    \catcode`\{=1
    \catcode`\}=2
    \@myverb%
}
\def\@myverb#1{%
    \endgroup%
    \texttt{#1}%
}
\newcommand{\scm}[3]{%
  % \vspace{2.1ex}
  \noindent
  \def\tmpp{#1}%
  \def\tmp{#3}\fbox{%
  \begin{minipage}{\linewidth}\vspace{0.5ex}
  \begin{minipage}[t]{0.17\linewidth}
    \texttt{\expandafter\foo\meaning\tmpp}
  \end{minipage}
  \begin{minipage}[t]{0.37\linewidth}%
    #2
  \end{minipage}
  \begin{minipage}[t]{0.24\linewidth}%
    \texttt{\expandafter\foo\meaning\tmp}
  \end{minipage}
  \begin{minipage}[t]{0.20\linewidth}
    #3
  \end{minipage}\vspace{0.5ex}
  \end{minipage}}\newline\noindent
}
\makeatother
\def\foo#1>{}
\makeatletter
\newcommand{\showcasemath}{%
    \begingroup
    % deactivate special characters
    \let\do\@makeother
    \dospecials
    % change '{' and '}' back to normal
    \catcode`\{=1
    \catcode`\}=2
    \@myverb%
}
\makeatother

\title{\texttt{lazyeqn}\textnormal{ --- equation macros for \LaTeXe}}
\author{Onur Solmaz}
\begin{document}
\maketitle
\section{Symbols}

This package aims to make typing of symbols much easier by providing some macros
prefixing conventions for different math alphabets. For example to type in a
upright bold A, one can use the \verb+\VA+ macro which prefixes the letter A
with the macro character \verb+\+ and the letter \verb+V+, which has been chosen
as the convention for ``upright bold'' math symbols.

These conventions have been chosen to reduce the number of characters typed,
whereas also to make sure the equations stay readable in code. For example, one might guess
the letter B might stand for bold and I might stand for italic in some prefixes,
but the prefixes don't have to make sense for all alphabets.
For the list of all prefixes for
specific alphabets, please refer to Table~\ref{tb:math}.

Normal Greek letters are slanted in lowercase and upright in uppercase. To
obtain bold Greek letters, use the \verb+\B+ prefix, e.g. \verb+\BXi+ for
uppercase xi and \verb+\Bxi+ for lowercase xi.

Also, \verb+\vareps+ and \verb+\varthe+ macros are defined as shortcuts to \verb+\varepsilon+ and \verb+\vartheta+

For numbers, the written forms can be prefixed to obtain corresponding symbols,
like \verb+\Vone+ and \verb+\Izero+. Observe the samples in Table~\ref{tb:math}
for alphabets in which this is possible.


\begin{table}[htbp]
  \centering
\begin{tabular}{p{14em} l l l}
  \toprule
  Name & \LaTeX{} package & Prefix & Sample \\
  \midrule
  Italic letters & none & none & $ABCdef$ \\
  Upright letters & none & \verb+\U+ & $\UA\UB\UC\Ud\Ue\Uf$\\
  Italic bold letters & none & \verb+\B+ & $\BA\BB\BC\Bd\Be\Bf\Bone\Btwo\Bthree$ \\
  Upright bold letters & none & \verb+\V+ & $\VA\VB\VC\Vd\Ve\Vf\Vone\Vtwo\Vthree$ \\
  Calligraphic letters (uppercase only) & none & \verb+\C+ & $\CA\CB\CC$ \\
  Calligraphic bold letters (uppercase only) & none & \verb+\CB+ & $\CBA\CBB\CBC$ \\
  AMS blackboard bold (uppercase only) & \texttt{amsfonts} & \verb+\I+ & $\IA\IB\IC$ \\
  \texttt{bbold} blackboard bold & \texttt{bbold},\texttt{mathbbol} & \verb+\II+ & $\IIA\IIB\IIC\IId\IIe\IIf\IIone\IItwo\IIthree$ \\
  \texttt{bbm} blackboard bold sans & \texttt{bbm} & \verb+\BB+ & $\BBA\BBB\BBC\BBd\BBe\BBf$ \\
  \texttt{bbm} blackboard bold serif & \texttt{bbm} & \verb+\bB+ & $\bBA\bBB\bBC\bBd\bBe\bBf$\\
  \texttt{bbm} blackboard bold sans italic & \texttt{bbm} & \verb+\IB+ & $\IBA\IBB\IBC\IBd\IBe\IBf$\\
  \texttt{bbm} blackboard bold serif italic & \texttt{bbm} & \verb+\iB+ & $\iBA\iBB\iBC\iBd\iBe\iBf$\\
  Raph Smith’s Formal Script (uppercase only) & \texttt{rsfs} & \verb+\RS+ & $\RSA\RSB\RSC$\\
  Raph Smith’s Formal Script bold (uppercase only) & \texttt{rsfs} & \verb+\BS+ & $\BSA\BSB\BSC$\\
  \texttt{euscript} cursive (uppercase only) & \texttt{euscript} & \verb+\eul+ & $\eulA\eulB\eulC$\\
  \texttt{euscript} cursive bold (uppercase only) & \texttt{euscript} & \verb+\Beul+ & $\BeulA\BeulB\BeulC$\\
  Fraktur & \texttt{yfonts} & \verb+\frak+ & $\frakA\frakB\frakC\fraka\frakb\frakc$\\
  Fraktur bold & \texttt{yfonts} & \verb+\Bfrak+ & $\BfrakA\BfrakB\BfrakC\Bfraka\Bfrakb\Bfrakc$\\
  \bottomrule
\end{tabular}
\caption{Guide for symbol prefixes for latin letters. For more information about math alphabets,
  please refer to {\em Table 213 -- Math Alphabets} of The
  Comprehensive \LaTeX{} Symbol List by Scott Pakin.}
\label{tb:math}
\end{table}

\section{Macros}

\subsection{Braces}

\scm{\rbr}{Round braces}{$\rbr{\VA}$}
\scm{\sbr}{Square brackets}{$\sbr{\VA}$}
\scm{\cbr}{Curly braces}{$\cbr{\VA}$}
\scm{\abr}{Angles}{$\abr{\VA}$}
\scm{\dbr}{Double square braces}{$\dbr{\VA}$} % maybe use rrbracket-llbracket from stmaryd?
\scm{\dbrl}{Left double square bracket}{$\dbrl$}
\scm{\dbrr}{Right double square bracket}{$\dbrr$}
\subsection{Operators}

% \begin{table}[H]
%   \centering
%   \begin{tabular}{l l l l}
%     \toprule
%     Macro & Explanation & Example usage & Result \\
%     \midrule
%     \verb+\tra+ & Transpose & &  \\
%     \verb+\variation{}{}+ & Variation & \verb+\variation{a}{b}+ & $\variation{a}{b}$  \\
%     \verb+\partd{}{}+ & Partial derivative & & \\
%     \verb+\partdb{}{}+ & Partial derivative with brackets & & \\
%     \verb+\cpartd{}{}+ & Partial derivative using \verb+\cfrac+ & & \\
%     \bottomrule
%   \end{tabular}
% \end{table}

% \scm{Macro}{Explanation}{Example usage}
%
\scm{\tra}{Transpose}{$\tra$}
\scm{\tra[]}{Transpose with preceding argument}{$\tra[-]$}
\scm{\tratop{}}{Transpose with argument on top}{$\tratop{23}$}
\scm{\variation}{Variation}{$\variation{a}{b}$}
\scm{\partd}{Partial derivative}{$\partd{a}{b}$}
\scm{\partdrb}{Partial derivative, round braces}{$\partdrb{a}{b}$}
\scm{\partdsb}{Partial derivative, square brackets}{$\partdsb{a}{b}$}
\scm{\cpartd}{Partial derivative with \texttt{cfrac}}{$\cpartd{a}{b}$}
\scm{\deriv}{Derivative}{$\deriv{a}{b}$}
\scm{\assem}{Assembly symbol}{$\assem$}
\scm{\evat}{Evaluated at}{$a\evat_{b=0}$}
\scm{\degree}{Degree symbol}{\degree{}C}
\scm{\linedot}{Time integral with overline}{$\linedot{\UA}$}
\scm{\DIV}{DIVergence}{$\DIV\VA$}
\scm{\Div}{DIVergence}{$\Div\VA$}
\scm{\div}{Divergence}{$\div\VA$}
\scm{\curl}{Curl}{$\curl\VA$}
\scm{\rot}{Rotation}{$\rot\VA$}
\scm{\proj}{Projection}{$\proj\VA$}
\scm{\sign}{Sign function}{$\sign\VA$}
\scm{\unit}{Unit function}{$\unit\VA$}
\scm{\arg}{arg}{$\arg\VA$}
\scm{\Arg}{Arg}{$\Arg\VA$}
\scm{\sym}{Symmetric}{$\sym\VA$}
\scm{\TR}{TRace}{$\TR\VA$}
\scm{\tr}{trace}{$\tr\VA$}
\scm{\trial}{Trial function}{$\trial\VA$}
\scm{\DEV}{DEViatoric}{$\DEV\VA$}
\scm{\dev}{deviatoric}{$\dev\VA$}
\scm{\GRAD}{GRADient}{$\GRAD\VA$}
\scm{\Grad}{Gradient}{$\Grad\VA$}
\scm{\grad}{gradient}{$\grad\VA$}
\scm{\Lin}{Linearization}{$\Lin\VA$}
\scm{\lin}{linearization}{$\lin\VA$}
\scm{\skw}{Skew}{$\skw\VA$}
\scm{\loc}{loc}{$\loc\VA$}
\scm{\cof}{Cofactor}{$\cof\VA$}
\scm{\reac}{reac}{$\reac\VA$}
\scm{\pind}{pind}{$\pind\VA$}
\scm{\ndim}{ndim}{$\ndim$}
\scm{\diag}{diag}{$\diag\VA$}
\scm{\veci{}}{Indexed list representation}{$\veci{a}$}
\scm{\dV}{Volume element}{$\dV$}
\scm{\dv}{Volume element}{$\dv$}
\scm{\dA}{Area element}{$\dA$}
\scm{\da}{Area element}{$\da$}
\scm{\dS}{Surface element}{$\dS$}
\scm{\dx}{Line element}{$\dx$}
\scm{\dt}{Time difference}{$\dt$}
\scm{\Abs{}}{Absolute value}{$\Abs{\VA}$}
\scm{\Norm{}}{Norm}{$\Norm{\VA}$}
\scm{\basis{}}{Vector bases}{$\basis{ijkl}$}
\scm{\rbdot}{Operands}{$\div\rbdot$}
\scm{\ddt}{Time derivative}{$\ddt x$}
\scm{\ddelt}{Partial Time derivative}{$\ddelt x$}
\scm{\so}{Special orthogonal group}{$\so$}
% \scm{\rve}{}{$\rve$}
% \scm{\rpg}{}{$\rpg$}
% \scm{\ehs}{}{$\ehs$}
% \scm{\mo}{}{$\mo$}
\scm{\othree}{Orthogonal group 3}{$\othree$}
\scm{\sothree}{Special orthogonal group 3}{$\sothree$}
\scm{\slthree}{Special linear group 3}{$\slthree$}
\scm{\AND}{and between equations}{$a \AND b$}
\scm{\WITH}{with between equations}{$a \WITH b$}
\scm{\com}{Comma between equations}{$a\com b$}

\subsection{Fractions and roots}
\scm{\half}{Half}{$\half$}
\scm{\third}{Third}{$\third$}
\scm{\fourth}{Fourth}{$\fourth$}
\scm{\fifth}{Fifth}{$\fifth$}
\scm{\sixth}{Sixth}{$\sixth$}
\scm{\seventh}{Seventh}{$\seventh$}
\scm{\eighth}{Eighth}{$\eighth$}
\scm{\ninth}{Ninth}{$\ninth$}
\scm{\tenth}{Tenth}{$\tenth$}
\scm{\twothree}{Twothree}{$\twothree$}
\scm{\threetwo}{Threetwo}{$\threetwo$}
\scm{\sqonetwo}{Sqonetwo}{$\sqonetwo$}
\scm{\sqonethree}{Sqonethree}{$\sqonethree$}
\scm{\sqtwothree}{Sqtwothree}{$\sqtwothree$}
\scm{\sqthreetwo}{Sqthreetwo}{$\sqthreetwo$}

\subsection{Vectors}
\scm{\cvec{}}{Column vector}{$\cvec{1;2;3;4}$}
\scm{\vcvec{}}{Column vector with straight braces}{$\vcvec{1;2;3;4}$}
\scm{\rvec{}}{Row vector}{$\rvec{1;2;3;4}$}
\scm{\vrvec{}}{Row vector with straight braces}{$\vrvec{1;2;3;4}$}

\end{document}
